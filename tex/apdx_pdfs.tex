\chapter{The Exponential Function as a PDF}
The normalized exponential function $f(x|k)$ has one free parameter $k$
\begin{equation*}
    f(x|k) = \mathcal{N} \mathrm{e}^{-kx} \,,
\end{equation*}
given a sufficient normalization factor $\mathcal{N}$.
For a given normalization interval $[a_1,a_2]$ parts of the normalization factor $\mathcal{N}$ can be moved into the argument of the exponential function
\begin{equation}
    \label{eq:apdx_pdfs_exp}
    f(x|k) = \frac{k \, \mathrm{e}^{-k(x-a_1)}}{1 - \mathrm{e}^{-k(a_2-a_1)}} \,.
\end{equation}
In the limit $k \to 0$ Eq.~\eqref{eq:apdx_pdfs_exp} is well defined
\begin{equation*}
    \lim_{k \to 0} f(x|k) = 1,
\end{equation*}
but numerically unstable.
Math libraries typically implement the relative error exponential function 
\begin{equation*}
    \texttt{exprel}(x) := \frac{\mathrm{e}^x - 1}{x} \,,
\end{equation*}
to avoid the loss of precision that occurs when $x$ is near zero.
Using this definition of \texttt{exprel} Eq.~\eqref{eq:apdx_pdfs_exp} can be rewritten
\begin{equation}
    \label{eq:apdx_pdfs_exprel}
    f(x|k) = \frac{\mathrm{e}^{-k(x-a_1)}}{(a_2-a_1) \, \texttt{exprel}(-k(a_2-a_1))} \,,
\end{equation}
and now allows sign flips of $k$ during the fitting process.

Fitting an exponential function on a single interval can easily be generalized for piecewise fits on two disjoint normalization intervals $[a_1,a_2]$ and $[b_1,b_2]$.
Doing so changes the normalization factor and Eq.~\eqref{eq:apdx_pdfs_exprel} becomes
\begin{equation}
    \label{eq:apdx_pdfs_exprel2}
    f(x|k) = \frac{\mathrm{e}^{-k(x-a_1)}}{(a_2-a_1) \, \texttt{exprel}(-k(a_2-a_1)) + (b_2 - b_1) \, \mathrm{e}^{-k(b_1 - a_1)} \, \texttt{exprel}(-k(b_2-b_1))} \,.
\end{equation}
In both cases the integral in $x$ can be written as
\begin{equation}
    \label{eq:apdx_pdfs_exprelint}
    \int_{x_1}^{x_2} \! \mathrm{d}x \, f(x|k) = (x_2 - x_1) \, \texttt{exprel}\left(-k (x_2 - x_1) \right) \times f(x_1|k) \,.
\end{equation}
This relation is useful when a fit for $k$ is used to inter- or extrapolate yields in given regions.
An uncertainty approximation by applying ordinary error propagation needs the derivative w.r.t.\ $k$ of Eq.~\eqref{eq:apdx_pdfs_exprelint} which is cluttered and again suffers from numerical instabilities.
When error propagation is needed, we will therefore revert to a numerical approach.
Eq.~\eqref{eq:apdx_pdfs_exprelint} is then evaluated for Gaussian distributed values of $k \sim \mathcal{G}(\mu,\sigma)$, where $\mu$ and $\sigma$ are the results of the preceding fit, and sorted.
The interval that spans a fraction of $\operatorname{erf}(1 / \sqrt 2) \approx 68\,\%$ of the sorted values, centered at the median is then taken as an approximation of the uncertainty interval.
