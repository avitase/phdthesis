\makeglossaries

\newglossaryentry{tevatron}{%
    name=Tevatron,
    description={Circular \proton \antiproton collider at Fermilab operating at a beam energy of $1\,$\tev}}


\newglossaryentry{ads}{%
    name={ADS},
    description={Method for extracting the \gls{ckm} angle $\gamma$ in decays such as \decay{\Lb}{\PD\Lz}, first proposed by Atwood, Dunietz and Soni~\cite{ads1,ads2}, \cf{} Sec.~\ref{sec:lbcpv}}}

\newglossaryentry{babar}{%
    name=Babar,
    description={Collaboration and \gls{hep} experiment at \gls{pepii}}}

\newglossaryentry{bdt}{%
    name={BDT},
    description={(Gradient) \textit{\underline{B}oosted \underline{D}ecision \underline{T}ree}, supervised (machine) learning model}}

\newglossaryentry{belle}{%
    name={Belle},
    description={Collaboration and \gls{hep} experiment at the \gls{kekb} accelerator}}

\newglossaryentry{belletwo}{%
    name={Belle II},
    description={Collaboration and \gls{hep} experiment at the \gls{superkekb} accelerator}}

\newglossaryentry{bepcii}{%
    name=BEPC II,
    description={Circular \ep\en Collider, operating at $\sqrt s = 2 \text{ to } 4.63\,$\gev}}

\newglossaryentry{besiii}{%
    name=BESIII,
    description={Collaboration and \gls{hep} experiment at the Beijing \ep\en Collider II (\gls{bepcii})}}

\newglossaryentry{cdf}{%
    name=CDF,
    description={Collaboration and \gls{hep} experiment at \gls{tevatron}}}

\newglossaryentry{ckm}{%
    name=CKM,
    description={Unitary matrix which contains information on the magnitudes and complex phases for flavor-changing weak interactions}}

\newglossaryentry{dchisqip}{%
    name={\ensuremath{\Delta \chi^2_\mathrm{IP}}},
    description={Difference between the $\chi^2$ value of the \gls{pv} reconstructed with and without the track under consideration}}

\newglossaryentry{DD}{%
    name=DD,
    description={Both final state particles of a $\PV^0$ decay (\eg{}, \decay{\Lz}{\proton\pim} or \decay{\KS}{\pip\pim}) are reconstructed as \textit{downstream tracks}, \cf{} Sec.~\ref{sec:detector_tracking} for more details}}

\newglossaryentry{dira}{%
    name=DIRA,
    description={\textit{\underline{Dir}ection \underline{a}ngle}, cosine of the angle between the momentum of the particle and the direction vector from some reference vertex or 3D-point to the end-vertex of the particle}}

\newglossaryentry{dll}{%
    name=DLL,
    description={\textit{\underline{D}elta \underline{l}og-\underline{l}ikelihood}}}

\newglossaryentry{doca}{%
    name=DOCA,
    description={\textit{\underline{D}istance \underline{o}f \underline{c}losest \underline{a}pproach}}}

\newglossaryentry{dof}{%
    name=DoF,
    description={\textit{\underline{D}egrees \underline{o}f \underline{F}reedom}}}

\newglossaryentry{dzero}{%
    name=\ensuremath{\text{D}\slashed{0}},
    description={Collaboration and \gls{hep} experiment at \gls{tevatron}}}

\newglossaryentry{dtf}{%
    name=DTF,
    description={\textit{\underline{D}ecay \underline{t}ree \underline{f}it}}}

\newglossaryentry{fn}{%
    name={FN},
    description={\textit{\underline{F}alse \underline{N}egative}, \cf{}~Appx.~\ref{chap:confmat}}}

\newglossaryentry{fnr}{%
    name={FNR},
    description={\textit{\underline{F}alse \underline{N}egative \underline{R}ate}, \cf{}~Appx.~\ref{chap:confmat}}}

\newglossaryentry{fom}{%
    name=FoM,
    description={\textit{\underline{F}igure \underline{o}f \underline{m}erit}}}

\newglossaryentry{fp}{%
    name={FP},
    description={\textit{\underline{F}alse \underline{P}ositve}, \cf{}~Appx.~\ref{chap:confmat}}}

\newglossaryentry{fpr}{%
    name={FPR},
    description={\textit{\underline{F}alse \underline{P}ositve \underline{R}ate}, \cf{}~Appx.~\ref{chap:confmat}}}

\newglossaryentry{ghostprob}{%
    name={Ghost Prob.},
    description={Output of an algorithm to identify tracks which do not correspond to the trajectory of a (single) true particle but rather originate from detector noise or multiple particles due to mismatching}}

\newglossaryentry{glw}{%
    name={GLW},
    description={Method for extracting the \gls{ckm} angle $\gamma$ in decays such as \decay{\Lb}{\PD\Lz}, first proposed by Gronau, London and Wyler~\cite{glw1,glw2}, \cf{} Sec.~\ref{sec:lbcpv}}}

\newglossaryentry{hep}{%
    name=HEP,
    description={\textit{\underline{H}igh \underline{E}nergy \underline{P}hysics}}}

\newglossaryentry{hlt}{%
    name=HLT,
    description={\textit{\underline{H}igh \underline{l}evel \underline{t}rigger}, includes a full off-line reconstruction}}

\newglossaryentry{kekb}{%
    name={KEKB},
    description={Circular \ep\en collider, operating at $\sqrt s = m(\FourS) = 10.57\,\gevcc$}}

\newglossaryentry{LL}{%
    name=LL,
    description={Both final state particles of a $\PV^0$ decay (\eg{}, \decay{\Lz}{\proton\pim} or \decay{\KS}{\pip\pim}) are reconstructed as \textit{long tracks}, \cf{} Sec.~\ref{sec:detector_tracking} for more details}}

\newglossaryentry{lzero}{%
    name=L0,
    description={Level 0 trigger of the LHCb trigger system, implemented in hardware, followed by the high level triggers}}

\newglossaryentry{mc}{%
    name={MC},
    description={\textit{\underline{M}onte \underline{C}arlo}, refers to the used simulation strategy. The \proton\proton collisions are generated using PYTHIA~\cite{pythia} with a specific \lhcb configuration~\cite{lhcbpythia}. Decays of unstable particles are described by EvtGen~\cite{evtgen}, where final-state radiations is generated using PHOTOS~\cite{photos}. The interaction of the particles with the detector (and its response) are implemented using the Geant4 toolkit~\cite{geant41,geant42} as described in Ref.~\cite{geant43}}}

\newglossaryentry{mva}{%
    name={MVA},
    description={\textit{\underline{M}ulti \underline{V}ariate \underline{A}nalysis}}}

\newglossaryentry{pca}{%
    name={PCA},
    description={\textit{\underline{P}rincipal \underline{C}omponent \underline{A}nalysis}, unsupervised (machine) learning technique, \cf{}~Appdx.~\ref{chap:pca}}}

\newglossaryentry{pcc}{%
    name=PCC,
    description={\textit{\underline{P}earson \underline{c}orrelation \underline{c}oefficient}, measure of the linear correlation between two variables. For a given sample of paired data $\{(x_1, y_1), \ldots (x_n, y_n)\}$ the PCC is given by the estimates of the covariance and variances:
$$\mathrm{PCC} = \frac{\sum\limits_{i=0}^n (x_i - \overline x)(y_i - \overline y)}{\sqrt{\sum\limits_{i=0}^n (x_i - \overline x)^2} \, \sqrt{\sum\limits_{i=0}^n (y_i - \overline y)^2}},$$
where $\overline{x}$ ($\overline{y}$) is the sample mean of $x_i$ ($y_i$).
The denominator normalizes this expression, according to the Cauchy-Schwarz inequality, to the interval $[-1, +1]$, where $+1$ indicates a total positive linear correlation, $0$ no linear correlation and $-1$ a total negative linear correlation.
Higher order correlations are not reflected in the $\mathrm{PCC}$}}

\newglossaryentry{pdf}{%
    name={PDF},
    description={\textit{\underline{P}robability \underline{D}ensity \underline{F}unction}}}

\newglossaryentry{pdg}{%
    name=PDG,
    description={\textit{\underline{P}article \underline{D}ata \underline{G}roup}~\cite{pdg}, collaboration of particle physicists that compiles and reanalyzes published results}}

\newglossaryentry{pepii}{%
    name={PEP-II},
    description={Circular asymmetric-energy \ep\en collider, operating at $\sqrt s = 10.58\,\gevcc$}}

\newglossaryentry{pid}{%
    name=PID,
    description={\textit{\underline{P}article \underline{id}entification}}}

\newglossaryentry{pv}{%
    name=PV,
    description={\textit{\underline{P}rimary \underline{V}ertex}, origin vertex of first particle in most decay chains, typically created by \proton\proton-scattering close to the beam-pipe}}

\newglossaryentry{reflection}{%
    name=reflection,
    description={Background contribution due to misidentification of at least one particle}}

\newglossaryentry{rich}{%
    name=RICH,
    description={\textit{\underline{R}ing \underline{I}maging \underline{Ch}erenkov}, system of the \lhcb detector providing identification of charged particles}}

\newglossaryentry{roc}{%
    name={ROC},
    description={\textit{\underline{R}eceiver \underline{O}perating \underline{C}haracteristic} (curve), graphical plot of \gls{tpr} vs.\ \gls{tnr} for varying threshold}}

\newglossaryentry{rocauc}{%
    name={ROC-AUC},
    description={\textit{\underline{A}rea \underline{U}nder \underline{ROC} \underline{C}urve}, probability that a given classifier will rank a randomly chosen positive instance higher than a randomly chosen negative one}}

\newglossaryentry{runone}{%
    name={Run~1},
    description={First major periode of data taking at the LHC, corresponding to $\mathcal{L} \approx 1\,\invfb$ at $\sqrt s = 7\,\tev$ (2011) and $\mathcal{L} \approx 2\,\invfb$ at $\sqrt s = 8\,\tev$ (2012)}}

\newglossaryentry{runtwo}{%
    name={Run~2},
    description={Second major periode of data taking at the LHC, corresponding to $\mathcal{L} \approx 6\,\invfb$ and $\sqrt s = 13\,\tev$}}

\newglossaryentry{stripping}{%
    name=stripping,
    description={Set of selection criteria applied to data directly after data taking}}

\newglossaryentry{superkekb}{%
    name=SuperKEKB,
    description={Circular \ep\en collider, operating at $\sqrt s = m(\FourS) = 10.57\,\gevcc$}}

\newglossaryentry{svm}{%
    name={SVM},
    description={\textit{\underline{S}upport-\underline{V}ector \underline{M}achine}, supervised (machine) learning model}}


\newglossaryentry{tis}{%
    name=TIS,
    description={\textit{\underline{t}rigger \underline{i}ndependent of \underline{s}ignal}}}

\newglossaryentry{tn}{%
    name={TN},
    description={\textit{\underline{T}rue \underline{N}egative}, \cf{}~Appx.~\ref{chap:confmat}}}

\newglossaryentry{tnr}{%
    name={TNR},
    description={\textit{\underline{T}rue \underline{N}egative \underline{R}ate}, \cf{}~Appx.~\ref{chap:confmat}}}

\newglossaryentry{tos}{%
    name=TOS,
    description={\textit{\underline{t}rigger \underline{o}n \underline{s}ignal}}}

\newglossaryentry{tp}{%
    name={TP},
    description={\textit{\underline{T}rue \underline{P}ositve}, \cf{}~Appx.~\ref{chap:confmat}}}

\newglossaryentry{tpr}{%
    name={TPR},
    description={\textit{\underline{T}rue \underline{P}ositve \underline{R}ate}, \cf{}~Appx.~\ref{chap:confmat}}}

\newglossaryentry{truthmatched}{%
    name={truth-matched},
    description={Each simulated track is attached with a label that allows a matching of tracks and simulated decays. If the reconstruction hypothesis of a candidate matches the \textit{true} generated one (including all tracks), the given candidate is referred to as \textit{truth-matched}}}

\newglossaryentry{truthmatching}{%
    name={truth-matching},
    description={Process of deciding whether a given simulated event is \gls{truthmatched} or \textit{unmatched}}}

\newglossaryentry{velo}{%
    name=VELO,
    description={\textit{\underline{Ve}rtex \underline{Lo}cator}, silicon vertex detector surrounding the interaction region}}

\newglossaryentry{wolfenstein}{%
    name={Wolfenstein parameterization},
    description={Parametrization of the \gls{ckm} matrix, introduced by Lincoln Wolfenstein with the four real parameters $\lambda$, $A$, $\rho$ and $\eta$}}
