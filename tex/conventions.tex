\chapter*{Conventions}
After spending years in the field of experimental high energy physics, adopting a certain kind of jargon and unspoken conventions is unavoidable.
The author tried his best to avoid jargon when possible and summarize abbreviations in a glossary on page \pageref{glossary}.

%\section*{Invariant Masses and Candidates}
Throughout this work we will frequently refer to invariant masses, \eg{}, as $m(\Lb)$ or $m(\Dz\Lz)$.
Whereas the former typically refers to the measured mass of the given hadron (the \Lb baryon in this example), the latter notation is meant as an abbreviation:
It should be read as the result of summing the invariant masses of a \Dz candidate and a \Lz candidate, according to four-momentum addition where \textit{candidates} are themselves recursively obtained by such four-momentum additions.
We note that in particular for recorded data, this does not only include the desired decay channel (\decay{\Lb}{\Dz\Lz} in the above example), but potentially also include random track combinations and other kind of background contributions.
On top of this, our notation of invariant masses also always implicitly includes the respective \CP conjugated particles, \eg{}, the invariant mass distribution $m(\Dz\Lz)$ is the result of a four-momentum addition of \Dz and \Lz candidates (\eg{}, \decay{\Lb}{\Dz\Lz}), as well as \Dzb and \Lbar candidates (\eg{}, \decay{\Lbbar}{\Dzb\Lbar}).

%\section*{Rounding}
%All numbers in this document that were obtained during the analysis and are given with respective uncertainties, are subject to a set of rounding rules.
%We respect the rounding rules stated by the Particle Data Group~\cite{pdg}:
%\begin{quotation}
%    The basic rule states that if the three highest order digits of the error lie between $100$ and $354$, we round to two significant digits.
%    If they lie between $355$ and $949$, we round to one significant digit.
%    Finally, if they lie between $950$ and $999$, we round up to $1000$ and keep two significant digits.
%    In all cases, the central value is given with a precision that matches that of the error.
%    So, for example, the result (coming from an average) $0.827 \pm 0.119$ would appear as $0.83 \pm 0.12$, while $0.827 \pm 0.367$ would turn into $0.8 \pm 0.4$.
%\end{quotation}
%Numbers that were taken from publications are not altered and thus are not necessarily subject of these rules.
