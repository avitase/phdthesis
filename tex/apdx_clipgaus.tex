\chapter{Clipped Gaussian Distribution}
\label{chap:apdx_clipgaus}
A large part of the free parameters of the fit model that we establish in Chap.~\ref{chap:fit} are fractions, such as $f_1$ and $f_s$, which are physically constrained to the interval $[0,1]$.
These boundaries enhance the numerical stability of the fit and makes it easy to separate or disable components of the likelihood by fixing the fractions to zero or one.
By design, $f \in [0,1]$ guarantees $0 \le \mathcal{L} \le 1$ and forbids negative signal yields which are meaningless for the present analysis.
As a consequence, the nominal values of fitted yields $f$ which can be assumed to be Gaussian distributed, will accumulate excessively at $f=0$, \ie{}, the assumed Gaussian distribution
\begin{equation*}
    \mathcal{G}(x|\mu,\sigma) = \frac{1}{\sqrt{2 \pi \sigma^2}} \, \mathrm{e}^{-\frac{1}{2} \left( \frac{x - \mu}{\sigma} \right)^2},
\end{equation*}
transforms, since $x$ is not distributed uniformly $\phi(x)=x$, but rather clipped at negative values $\phi(x) = \Theta(x) \, x$,
\begin{equation*}
    \mathcal{G}(x|\mu, \sigma) \; \xmapsto{\phi(x) = \Theta(x) \, x} \; \frac{\delta(x)}{2} \operatorname{erfc} \frac{\mu}{\sqrt{2}\sigma} + \mathcal{G}(x|\mu, \sigma) =: \tilde{\mathcal{G}}(x|\mu, \sigma) \,.
\end{equation*}
This clipping changes the moments non-trivially, in particular,
\begin{align}
    \langle x \rangle_{\tilde{\mathcal{G}}} &= \mu \left[ 1- \frac{1}{2} \operatorname{erfc} \frac{\mu}{\sqrt{2}\sigma} \right] + \frac{\sigma}{\sqrt{2 \pi}} \mathrm{e}^{-\frac{1}{2} \left( \frac{\mu}{\sigma} \right)^2} \neq \mu = \langle x \rangle_{\mathcal{G}} \,, \label{eq:apdx_clipgaus_mean} \\
    \langle x^2 \rangle_{\tilde{\mathcal{G}}} &= \left( \mu^2 + \sigma^2 \right) \left[ 1- \frac{1}{2} \operatorname{erfc} \frac{\mu}{\sqrt{2}\sigma} \right] + \frac{\mu \sigma}{\sqrt{2 \pi}} \mathrm{e}^{-\frac{1}{2} \left( \frac{\mu}{\sigma} \right)^2} \neq \mu^2 + \sigma^2 = \langle x^2 \rangle_{\mathcal{G}} \,. \label{eq:apdx_clipgaus_std}
\end{align}
We note that, although the mean value is shifted, this does not contribute to a potential bias of the fit if the genuine yield (fraction) is non-zero, since it does not affect all values, but only unphysical $f<0$ fluctuations.

If instead considering $\phi(x) = \Theta(x) \, x^2$ which would corresponds to a $\chi_1^2$-distribution with one \gls{dof} in the unclipped case with $\mu=0$ and $\sigma=1$, the distribution now becomes a clipped $\chi^2$-distribution:
\begin{equation*}
    \mathcal{G}(x|0, 1) \; \xmapsto{\phi(x) = \Theta(x) \, x^2} \; \frac{1 + \delta(x)}{2} \frac{\exp(-x/2)}{\sqrt{2 \pi x}} = \frac{1 + \delta(x)}{2} \chi_1^2(x) =: \tilde \chi_1^2(x) \,.
\end{equation*}
The mean and variances of $\tilde \chi_1^2(x)$ are
\begin{align}
    \langle x \rangle_{\tilde\chi_1^2} &= \frac{1}{2} \langle x \rangle_{\chi_1^2} = \frac{1}{2} \,, \label{eq:apdx_clipchisq_mean} \\
    \langle x^2 \rangle_{\tilde\chi_1^2} - \langle x \rangle_{\tilde\chi_1^2}^2 &= \frac{1}{2} \langle x^2 \rangle_{\chi_1^2} - \frac{1}{4} \langle x \rangle_{\chi_1^2}^2 = \frac{3}{2} - \frac{1}{4} = \frac{5}{4} \,. \label{eq:apdx_clipchisq_std}
\end{align}
