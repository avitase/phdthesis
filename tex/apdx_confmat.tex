\chapter{Confusion Matrix}
\label{chap:confmat}
Our definitions of \textit{true} and \textit{false}, as well as \textit{positive} and \textit{negative} in the context of classification throughout the present analysis are listed in the confusion matrix, shown in Fig.~\ref{fig:confmat}.
The matrix reads as following: a row listed the predicted class label and a column shows the genuine class label, \eg{}, we refer to events that are predicted being of class \textit{background}, but are actually genuine \textit{signal} events as \gls{fn}, whereas we refer to genuine background events, that are spuriously classified as \textit{signal}, as \gls{fp}.

\begin{figure}[htbp]
    \centering
    \setlength\unitlength{.8cm}
    \begin{picture}(4,5.5)
        \multiput(0,0)(0,2){3}{\line(1,0){4}}
        \multiput(0,0)(2,0){3}{\line(0,1){4}}

        \put(1,1){\makebox(0,0){FN}}
        \put(3,1){\makebox(0,0){TN}}

        \put(1,3){\makebox(0,0){TP}}
        \put(3,3){\makebox(0,0){FP}}

        \put(-.7,3){\rotatebox[origin=c]{90}{Sig.}}
        \put(-.7,1){\rotatebox[origin=c]{90}{Bkg.}}

        \put(1,4.5){\makebox(0,0){Sig.}}
        \put(3,4.5){\makebox(0,0){Bkg.}}

        \put (-1.5,2){\rotatebox[origin=c]{90}{Predicted}}
        \put(2,5.3){\makebox(0,0){Actual}}
    \end{picture}
    \caption{Confusion matrix as used in the present analysis for the classes \textit{signal} (Sig.) and \textit{background} (Bkg.). The abbreviations \gls{tp}, \gls{fp}, \gls{fn} and \gls{tn} refer to \textit{true positive}, \textit{false positive}, \textit{false negative}, and \textit{true negative}, respectively.}
    \label{fig:confmat}
\end{figure}

