\chapter*{Introduction}
The idea of simplifying the description of different types of matter by introducing fundamental substructures is old.
In Ancient Greece the \textit{elements} earth, water, air and fire were assumed to be fundamental.
In the early 1800s, John Dalton used the concept of atoms to explain why elements always react in ratios of small whole numbers.
Later, in the early 1900s, Ernest Rutherford discovered the nucleus.
Today, the concept of protons, neutrons and electrons building different types of atoms is well established and taught in school.
The standard model of particle physics (SM) introduces a new layer of fundamental particles.
Particles like the proton and the neutron are hadrons, composed of quarks.
The electron stays fundamental and joins the group of the fundamental leptons.
There are more quarks than one finds within protons and neutrons, and there are more fundamental leptons than the electron.
However, these particles are not stable and decay until only the lightest particles are left, \ie{}, in nature we only observe protons, neutrons and electrons most of the time.
The proton is a baryon and the only stable hadron.
The neutron is a long-living particle but eventually decays via the $\beta$-decay,
\begin{equation*}
    \decay{\neutron}{\proton\en\neueb}.
\end{equation*}
Mesons are another possible combination of quarks.
Instead of consisting of three quarks (baryon), they are made of one quark and one anti-quark.
Whereas there is one stable baryon, all mesons are unstable.
The lightest mesons are the pion triplet (\pip, \piz, \pim) and their dominant decay modes are:
\begin{align*}
    \piz &\to 2\gamma, \\
    \pip &\to \mup \neumb, \\
    \pim &\to \mun \neum,
\end{align*}
where the final states of the charged (neutral) pions are leptons (bosons).

Classical atomic models do not need mesons, since their physics is dominated by the interaction between the protons, neutrons and electrons.
The lion's share of the visible matter in our universe is baryonic and not mesonic.
On the one hand, it is the \CP violation in meson decays that is well studied at colliders but, on the other hand, one of the unsolved mysteries of our universe is its large baryon anti-baryon asymmetry (and not meson anti-meson asymmetry) which can be considered as the largest known macroscopic \CP violation.
The SM of particle physics (as well as the SM of cosmology) yet have failed explaining the order of magnitude of this asymmetry, whereas at the same time results of \CP violation in meson decays are in great agreement with the SM predictions.

In the present analysis we contribute to the experimental foundation for studies of \CP violations in baryon decays by searching for two-body decays of the \Lb and the \Xibz baryon which can be used at future experiments to estimate the \CP violation parameter $\gamma$.
